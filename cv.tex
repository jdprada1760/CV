%------------------------------------
% Dario Taraborelli
% Typesetting your academic CV in LaTeX
%
% URL: http://nitens.org/taraborelli/cvtex
% DISCLAIMER: This template is provided for free and without any guarantee 
% that it will correctly compile on your system if you have a non-standard  
% configuration.
% Some rights reserved: http://creativecommons.org/licenses/by-sa/3.0/
%------------------------------------

%!TEX TS-program = xelatex
%!TEX encoding = UTF-8 Unicode

\documentclass[10pt, a4paper]{article}
\usepackage{fontspec} 
\usepackage{gensymb}

% DOCUMENT LAYOUT
\usepackage{geometry} 
\geometry{a4paper, textwidth=5.5in, textheight=9.5in, marginparsep=7pt, marginparwidth=.6in}
\setlength\parindent{0in}

% FONTS
\usepackage[usenames,dvipsnames]{xcolor}
\usepackage{xunicode}
\usepackage{xltxtra}
\defaultfontfeatures{Mapping=tex-text}
%\setromanfont [Ligatures={Common}, Numbers={OldStyle}, Variant=01]{Linux Libertine O}
%\setmonofont[Scale=0.8]{Monaco}
%%% modified by Karol Kozioł for ShareLaTeX use
\setmainfont[
  Ligatures={Common}, Numbers={OldStyle}, Variant=01,
  BoldFont=LinLibertine_RB.otf,
  ItalicFont=LinLibertine_RI.otf,
  BoldItalicFont=LinLibertine_RBI.otf
]{LinLibertine_R.otf}
\setmonofont[Scale=0.8]{DejaVuSansMono.ttf}

% ---- CUSTOM COMMANDS
\newcommand{\html}[1]{\href{#1}{\scriptsize\textsc{[html]}}}
\newcommand{\pdf}[1]{\href{#1}{\scriptsize\textsc{[pdf]}}}
\newcommand{\doi}[1]{\href{#1}{\scriptsize\textsc{[doi]}}}
% ---- MARGIN YEARS
\usepackage{marginnote}
\newcommand{\amper{}}{\chardef\amper="E0BD }
\newcommand{\years}[1]{\marginnote{\scriptsize #1}}
\renewcommand*{\raggedleftmarginnote}{}
\setlength{\marginparsep}{7pt}
\reversemarginpar

% HEADINGS
\usepackage{sectsty} 
\usepackage[normalem]{ulem} 
\sectionfont{\mdseries\upshape\Large}
\subsectionfont{\mdseries\scshape\normalsize} 
\subsubsectionfont{\mdseries\upshape\large} 

% PDF SETUP
% ---- FILL IN HERE THE DOC TITLE AND AUTHOR
\usepackage[%dvipdfm, 
bookmarks, colorlinks, breaklinks, 
% ---- FILL IN HERE THE TITLE AND AUTHOR
	pdftitle={CDS.CV},
	pdfauthor={Jesus David Prada Gonzalez},
]{hyperref}  
\hypersetup{linkcolor=blue,citecolor=blue,filecolor=black,urlcolor=MidnightBlue} 

% DOCUMENT
\begin{document}

{\LARGE Jesus David Prada Gonzalez}\\[1cm]
 Universidad de los Andes\\
Physics Department\\
Cra 1 N$^{\circ}$ 18A- 12, Bogota D.C.\\
Phone: \texttt{+57 3204737920}\\
email: \href{mailto:c.jd.prada1760@uniandes.edu.co}{jd.prada1760@uniandes.edu.co}\\
\textsc{url}: \href{https://github.com/jdprada1760/}{https://github.com/jdprada1760/}\\

Born:  March 8, 1995-Barrancabermeja, Colombia\\
Nationality: Colombian

%%\hrule
\section*{Current position}
\emph{Master of Sciences in Physics Student}, Universidad de los Andes

%%\hrule
\section*{Areas of specialization}
Computational Astrophysics • Galactic \& Extragalactic Astrophysics • Theoretical physics

%\hrule
\section*{Education}
\noindent
\years{2012-2016}\textsc{Bachelor of Sciences} in Physics, Universidad de los Andes. GPA 4.66/5.0 \\

\years{2016-Ongoing}\textsc{Master of Sciences} in Physics, Universidad de los Andes. GPA 4.77/5.0

\section*{Research}
\noindent
\years{Summer 2015} {\textbf{Research Experiences for Undergraduates (REU) program at Cornell University}}: Comparison of the Schechter parameters of Halo mass function in different Halo environments from the Millennium Simulation. Advisers: Martha Haynes, Michael Jones \& David Chernov.\\

\years{July \& \\ November 2017}{\textbf{Research internship at Heidelberg's Institute of Theoretical studies}}: The expected shape of the Milky Way's Dark Matter Halo. Advisers: Volker Springel \& Jaime Forero.\\

\years{Ongoing} {\textbf{First author publication}}: The influence of the environment on the HI mass functions in cosmological simulations. Collaborators: Martha Haynes, Michael Jones, Ricardo Giovanelly \& Jaime Forero.

\section*{Schools \& Events}
\years{Dec 2014} \textbf{Oral presentation at the Colombian Congress of Astronomy and Astrophysics (COCOA)}. Pasto, Colombia. Title: A Dark Matter density estimator that uses information from the phase space. Advisor: Jaime Forero.\\

\years{Oct 2016} \textbf{Oral presentation at the Latin American XV Regional IAU Meeting (LARIM)}. Cartagena, Colombia. Title: The influence of the environment on the HI mass functions in cosmological simulations. Collaborators: Martha Haynes, Michael Jones, Ricardo Giovanelly \& Jaime Forero.\\ 

\years{Oct 2017} \textbf{Poster at the Colombian Congress of Astronomy and Astrophysics (COCOA)}.  Pereira, Colombia. Title: The influence of the environment on the HI mass functions in cosmological simulations. Collaborators: Martha Haynes, Michael Jones, Ricardo Giovanelly \& Jaime Forero.

\section*{Computing \& Researcher Skills}

\textbf{Systems}: Linux, MSWindows\\

\textbf{Development}: C, Python, Bash, SQL, Java, OpenMP, MPI.\\ 

\textbf{Software}: \LaTeX, MATLAB, Mathematica.\\

\textbf{Tools}: Analysis of Arepo \& Gadget-2 output databases. Parallelization. Finite differences. Finite volume. Finite elements. Monte Carlo. Machine Learning.

\section*{Service to the profession}

\years{2013-2016}\textbf{Professor's assistant} for basic and advanced physics courses at Universidad de los Andes:
Physics II, Teaching Practice \& Analytical Mechanics\\

\years{2013-2015} \textbf{Teaching assistant} in Cl\'inica de Problemas (Problem solving assistance for undergraduate
students) at Universidad de los Andes.\\

\years{2016-II} \textbf{Teacher} of Experimental Physics I at Universidad de los Andes\\

\years{2017} \textbf{Teacher} of Computational Tools at Universidad de los Andes

\section*{Grants \& awards}
\noindent

\years{2012} \textsc{Academic Excellence Distinction 2012-I} for obtaining the best GPA among all students from the Physics undergraduate program at Universidad de los Andes

\section*{Languages}

\textsc{Spanish}: Native 

\textsc{English}: Fluent (TOEFL Test Score 102/120)

\section*{Personal Interests}

Speed Cubing (solving the Rubik’s Cube as fast as possible).\\
Science Fiction.\\
French \& German Languages.\\ 
Calisthenics.\\
Machine Learning.\\
Quantum Mechanics Foundations.\\

\end{document}